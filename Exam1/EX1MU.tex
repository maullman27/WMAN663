% Options for packages loaded elsewhere
\PassOptionsToPackage{unicode}{hyperref}
\PassOptionsToPackage{hyphens}{url}
%
\documentclass[
]{article}
\usepackage{lmodern}
\usepackage{amssymb,amsmath}
\usepackage{ifxetex,ifluatex}
\ifnum 0\ifxetex 1\fi\ifluatex 1\fi=0 % if pdftex
  \usepackage[T1]{fontenc}
  \usepackage[utf8]{inputenc}
  \usepackage{textcomp} % provide euro and other symbols
\else % if luatex or xetex
  \usepackage{unicode-math}
  \defaultfontfeatures{Scale=MatchLowercase}
  \defaultfontfeatures[\rmfamily]{Ligatures=TeX,Scale=1}
\fi
% Use upquote if available, for straight quotes in verbatim environments
\IfFileExists{upquote.sty}{\usepackage{upquote}}{}
\IfFileExists{microtype.sty}{% use microtype if available
  \usepackage[]{microtype}
  \UseMicrotypeSet[protrusion]{basicmath} % disable protrusion for tt fonts
}{}
\makeatletter
\@ifundefined{KOMAClassName}{% if non-KOMA class
  \IfFileExists{parskip.sty}{%
    \usepackage{parskip}
  }{% else
    \setlength{\parindent}{0pt}
    \setlength{\parskip}{6pt plus 2pt minus 1pt}}
}{% if KOMA class
  \KOMAoptions{parskip=half}}
\makeatother
\usepackage{xcolor}
\IfFileExists{xurl.sty}{\usepackage{xurl}}{} % add URL line breaks if available
\IfFileExists{bookmark.sty}{\usepackage{bookmark}}{\usepackage{hyperref}}
\hypersetup{
  pdftitle={Exam 1},
  pdfauthor={MacKenzie Ullman},
  hidelinks,
  pdfcreator={LaTeX via pandoc}}
\urlstyle{same} % disable monospaced font for URLs
\usepackage[margin=1in]{geometry}
\usepackage{color}
\usepackage{fancyvrb}
\newcommand{\VerbBar}{|}
\newcommand{\VERB}{\Verb[commandchars=\\\{\}]}
\DefineVerbatimEnvironment{Highlighting}{Verbatim}{commandchars=\\\{\}}
% Add ',fontsize=\small' for more characters per line
\usepackage{framed}
\definecolor{shadecolor}{RGB}{248,248,248}
\newenvironment{Shaded}{\begin{snugshade}}{\end{snugshade}}
\newcommand{\AlertTok}[1]{\textcolor[rgb]{0.94,0.16,0.16}{#1}}
\newcommand{\AnnotationTok}[1]{\textcolor[rgb]{0.56,0.35,0.01}{\textbf{\textit{#1}}}}
\newcommand{\AttributeTok}[1]{\textcolor[rgb]{0.77,0.63,0.00}{#1}}
\newcommand{\BaseNTok}[1]{\textcolor[rgb]{0.00,0.00,0.81}{#1}}
\newcommand{\BuiltInTok}[1]{#1}
\newcommand{\CharTok}[1]{\textcolor[rgb]{0.31,0.60,0.02}{#1}}
\newcommand{\CommentTok}[1]{\textcolor[rgb]{0.56,0.35,0.01}{\textit{#1}}}
\newcommand{\CommentVarTok}[1]{\textcolor[rgb]{0.56,0.35,0.01}{\textbf{\textit{#1}}}}
\newcommand{\ConstantTok}[1]{\textcolor[rgb]{0.00,0.00,0.00}{#1}}
\newcommand{\ControlFlowTok}[1]{\textcolor[rgb]{0.13,0.29,0.53}{\textbf{#1}}}
\newcommand{\DataTypeTok}[1]{\textcolor[rgb]{0.13,0.29,0.53}{#1}}
\newcommand{\DecValTok}[1]{\textcolor[rgb]{0.00,0.00,0.81}{#1}}
\newcommand{\DocumentationTok}[1]{\textcolor[rgb]{0.56,0.35,0.01}{\textbf{\textit{#1}}}}
\newcommand{\ErrorTok}[1]{\textcolor[rgb]{0.64,0.00,0.00}{\textbf{#1}}}
\newcommand{\ExtensionTok}[1]{#1}
\newcommand{\FloatTok}[1]{\textcolor[rgb]{0.00,0.00,0.81}{#1}}
\newcommand{\FunctionTok}[1]{\textcolor[rgb]{0.00,0.00,0.00}{#1}}
\newcommand{\ImportTok}[1]{#1}
\newcommand{\InformationTok}[1]{\textcolor[rgb]{0.56,0.35,0.01}{\textbf{\textit{#1}}}}
\newcommand{\KeywordTok}[1]{\textcolor[rgb]{0.13,0.29,0.53}{\textbf{#1}}}
\newcommand{\NormalTok}[1]{#1}
\newcommand{\OperatorTok}[1]{\textcolor[rgb]{0.81,0.36,0.00}{\textbf{#1}}}
\newcommand{\OtherTok}[1]{\textcolor[rgb]{0.56,0.35,0.01}{#1}}
\newcommand{\PreprocessorTok}[1]{\textcolor[rgb]{0.56,0.35,0.01}{\textit{#1}}}
\newcommand{\RegionMarkerTok}[1]{#1}
\newcommand{\SpecialCharTok}[1]{\textcolor[rgb]{0.00,0.00,0.00}{#1}}
\newcommand{\SpecialStringTok}[1]{\textcolor[rgb]{0.31,0.60,0.02}{#1}}
\newcommand{\StringTok}[1]{\textcolor[rgb]{0.31,0.60,0.02}{#1}}
\newcommand{\VariableTok}[1]{\textcolor[rgb]{0.00,0.00,0.00}{#1}}
\newcommand{\VerbatimStringTok}[1]{\textcolor[rgb]{0.31,0.60,0.02}{#1}}
\newcommand{\WarningTok}[1]{\textcolor[rgb]{0.56,0.35,0.01}{\textbf{\textit{#1}}}}
\usepackage{graphicx,grffile}
\makeatletter
\def\maxwidth{\ifdim\Gin@nat@width>\linewidth\linewidth\else\Gin@nat@width\fi}
\def\maxheight{\ifdim\Gin@nat@height>\textheight\textheight\else\Gin@nat@height\fi}
\makeatother
% Scale images if necessary, so that they will not overflow the page
% margins by default, and it is still possible to overwrite the defaults
% using explicit options in \includegraphics[width, height, ...]{}
\setkeys{Gin}{width=\maxwidth,height=\maxheight,keepaspectratio}
% Set default figure placement to htbp
\makeatletter
\def\fps@figure{htbp}
\makeatother
\setlength{\emergencystretch}{3em} % prevent overfull lines
\providecommand{\tightlist}{%
  \setlength{\itemsep}{0pt}\setlength{\parskip}{0pt}}
\setcounter{secnumdepth}{-\maxdimen} % remove section numbering

\title{Exam 1}
\author{MacKenzie Ullman}
\date{2/17/2021}

\begin{document}
\maketitle

\#1 Import this dataset into R and inspect the first several rows of
your data

\begin{Shaded}
\begin{Highlighting}[]
\KeywordTok{setwd}\NormalTok{(}\StringTok{"C:/Users/mau0005/Downloads"}\NormalTok{)}
\KeywordTok{getwd}\NormalTok{()}
\end{Highlighting}
\end{Shaded}

\begin{verbatim}
## [1] "C:/Users/mau0005/Downloads"
\end{verbatim}

\begin{Shaded}
\begin{Highlighting}[]
\NormalTok{Exam1<-}\KeywordTok{read.csv}\NormalTok{(}\DataTypeTok{file=}\StringTok{'Exam 1 Data.csv'}\NormalTok{)}
\KeywordTok{head}\NormalTok{(Exam1)}
\end{Highlighting}
\end{Shaded}

\begin{verbatim}
##            y          x1          x2 x3
## 1 -1.2583929 -0.78501384 -0.43328324  b
## 2  0.8453694 -1.64769643 -0.44504148  c
## 3 -0.3548917 -0.23241743 -0.98055862  b
## 4  0.1184083 -0.96179449 -1.43950151  c
## 5  2.3361615  0.38517298  0.41564078  c
## 6  0.9902994 -0.06704533 -0.05012007  b
\end{verbatim}

\#2 Fit a linear model that assumes your response is a function of x1,
x2, and x3. Include an interaction between x1 and x2 only (i.e., do not
include an interaction between your categorical variables and any other
variables).

\begin{Shaded}
\begin{Highlighting}[]
\NormalTok{fit<-}\KeywordTok{lm}\NormalTok{(y }\OperatorTok{~}\StringTok{ }\NormalTok{x1 }\OperatorTok{*}\StringTok{ }\NormalTok{x2 }\OperatorTok{+}\StringTok{ }\NormalTok{x3, }\DataTypeTok{data=}\NormalTok{ Exam1)}
\NormalTok{fit}
\end{Highlighting}
\end{Shaded}

\begin{verbatim}
## 
## Call:
## lm(formula = y ~ x1 * x2 + x3, data = Exam1)
## 
## Coefficients:
## (Intercept)           x1           x2          x3b          x3c        x1:x2  
##    0.701218     0.260469     0.283447    -1.627162     0.002504     0.149610
\end{verbatim}

\begin{Shaded}
\begin{Highlighting}[]
\KeywordTok{summary}\NormalTok{(fit)}
\end{Highlighting}
\end{Shaded}

\begin{verbatim}
## 
## Call:
## lm(formula = y ~ x1 * x2 + x3, data = Exam1)
## 
## Residuals:
##      Min       1Q   Median       3Q      Max 
## -2.88266 -0.68069  0.04797  0.73887  2.27856 
## 
## Coefficients:
##              Estimate Std. Error t value Pr(>|t|)    
## (Intercept)  0.701218   0.183975   3.811 0.000247 ***
## x1           0.260469   0.107963   2.413 0.017781 *  
## x2           0.283447   0.109977   2.577 0.011512 *  
## x3b         -1.627162   0.250457  -6.497 3.88e-09 ***
## x3c          0.002504   0.268673   0.009 0.992584    
## x1:x2        0.149610   0.089865   1.665 0.099279 .  
## ---
## Signif. codes:  0 '***' 0.001 '**' 0.01 '*' 0.05 '.' 0.1 ' ' 1
## 
## Residual standard error: 1.044 on 94 degrees of freedom
## Multiple R-squared:  0.4472, Adjusted R-squared:  0.4178 
## F-statistic: 15.21 on 5 and 94 DF,  p-value: 6.249e-11
\end{verbatim}

\#3 Interpret the effect of variable x1 when x2 = -1

\begin{Shaded}
\begin{Highlighting}[]
\NormalTok{b <-}\StringTok{ }\KeywordTok{coef}\NormalTok{(fit)}
\NormalTok{b}
\end{Highlighting}
\end{Shaded}

\begin{verbatim}
##  (Intercept)           x1           x2          x3b          x3c        x1:x2 
##  0.701217624  0.260468593  0.283447196 -1.627162195  0.002504032  0.149609715
\end{verbatim}

\begin{Shaded}
\begin{Highlighting}[]
\NormalTok{x2<-}\StringTok{ }\DecValTok{-1}
\NormalTok{x1<-}\StringTok{ }\NormalTok{b[}\DecValTok{2}\NormalTok{] }\OperatorTok{+}\NormalTok{b[}\DecValTok{6}\NormalTok{]}\OperatorTok{*}\StringTok{ }\DecValTok{-1}
\NormalTok{x1}
\end{Highlighting}
\end{Shaded}

\begin{verbatim}
##        x1 
## 0.1108589
\end{verbatim}

\#4 Interpret the effect of variable x1 when x2 = 1

\begin{Shaded}
\begin{Highlighting}[]
\NormalTok{x2<-}\StringTok{ }\DecValTok{1}
\NormalTok{x1<-}\StringTok{ }\NormalTok{b[}\DecValTok{2}\NormalTok{] }\OperatorTok{+}\StringTok{ }\NormalTok{b[}\DecValTok{6}\NormalTok{] }\OperatorTok{*}\StringTok{ }\DecValTok{1}
\NormalTok{x1}
\end{Highlighting}
\end{Shaded}

\begin{verbatim}
##        x1 
## 0.4100783
\end{verbatim}

\#5 Interpret the effect of variable x3

\begin{Shaded}
\begin{Highlighting}[]
\NormalTok{x3<-}\StringTok{ }\NormalTok{b[}\DecValTok{4}\NormalTok{] }\OperatorTok{+}\StringTok{ }\NormalTok{b[}\DecValTok{5}\NormalTok{]}
\NormalTok{x3}
\end{Highlighting}
\end{Shaded}

\begin{verbatim}
##       x3b 
## -1.624658
\end{verbatim}

\#6 Describe how R codes the categorical variable x3. Demonstrate by
reporting the first 5 values of variables derived from x3

R creates k − 1 dummy variables, where k is the number of levels of
categorical variables. So here, R would create 2 dummy variables. A is
set aside as the reference, so there is a dummy variable associated with
variable b that equals 1 if a factor level is b and equals 0 otherwise.
Similarly, there is another variable c that equals 1 if a factor level
is c and equals 0 otherwise.

\begin{Shaded}
\begin{Highlighting}[]
\KeywordTok{cbind}\NormalTok{(Exam1}\OperatorTok{$}\NormalTok{x3[}\DecValTok{1}\OperatorTok{:}\DecValTok{5}\NormalTok{],}
\KeywordTok{ifelse}\NormalTok{(Exam1}\OperatorTok{$}\NormalTok{x3 }\OperatorTok{==}\StringTok{ 'b'}\NormalTok{, }\DecValTok{1}\NormalTok{, }\DecValTok{0}\NormalTok{)[}\DecValTok{1}\OperatorTok{:}\DecValTok{5}\NormalTok{],}
\KeywordTok{ifelse}\NormalTok{(Exam1}\OperatorTok{$}\NormalTok{x3 }\OperatorTok{==}\StringTok{ 'c'}\NormalTok{, }\DecValTok{1}\NormalTok{, }\DecValTok{0}\NormalTok{)[}\DecValTok{1}\OperatorTok{:}\DecValTok{5}\NormalTok{])}
\end{Highlighting}
\end{Shaded}

\begin{verbatim}
##      [,1] [,2] [,3]
## [1,] "b"  "1"  "0" 
## [2,] "c"  "0"  "1" 
## [3,] "b"  "1"  "0" 
## [4,] "c"  "0"  "1" 
## [5,] "c"  "0"  "1"
\end{verbatim}

\#7 Derive the test statistic and p-value associated with the
interaction between x1 and x2. What is the null hypothesis assumed by
the ``lm()'' function? Do we reject or fail to reject this null
hypothesis? Defend your answer.

\begin{Shaded}
\begin{Highlighting}[]
\CommentTok{# test stat}
\NormalTok{ts <-}\StringTok{ }\KeywordTok{coef}\NormalTok{(fit)[}\DecValTok{6}\NormalTok{]}\OperatorTok{/}\KeywordTok{summary}\NormalTok{(fit)[[}\StringTok{'coefficients'}\NormalTok{]][}\DecValTok{6}\NormalTok{, }\DecValTok{3}\NormalTok{]}
\NormalTok{ts; }\KeywordTok{summary}\NormalTok{(fit)[[}\StringTok{'coefficients'}\NormalTok{]][}\DecValTok{6}\NormalTok{, }\DecValTok{4}\NormalTok{]}
\end{Highlighting}
\end{Shaded}

\begin{verbatim}
##      x1:x2 
## 0.08986547
\end{verbatim}

\begin{verbatim}
## [1] 0.09927881
\end{verbatim}

\begin{Shaded}
\begin{Highlighting}[]
\CommentTok{# pvalue}
\KeywordTok{pt}\NormalTok{(ts, }\DataTypeTok{df =} \KeywordTok{nrow}\NormalTok{(Exam1) }\OperatorTok{-}\StringTok{ }\KeywordTok{length}\NormalTok{(}\KeywordTok{coef}\NormalTok{(fit))) }\OperatorTok{*}\StringTok{ }\DecValTok{2}
\end{Highlighting}
\end{Shaded}

\begin{verbatim}
##    x1:x2 
## 1.071415
\end{verbatim}

The null hypothesis is that the slope coefficient associated with the
interaction between x1 and x2 is 0. We do not reject the null hypothesis
because the p-value is larger than any reasonable α.

\#8 assume you have the following realizations of random variable Y :y =
(3, 8, 7) Further assume realizations of the random variable Y are
Gaussian distributed: y ∼ Gaussian(µ, σ2). Fix σ 2 = 1 and µ = 8, and
evaluate the probability density at each of your 3 realizations.

\begin{Shaded}
\begin{Highlighting}[]
\NormalTok{y <-}\StringTok{ }\KeywordTok{c}\NormalTok{(}\DecValTok{3}\NormalTok{,}\DecValTok{8}\NormalTok{,}\DecValTok{7}\NormalTok{)}
\KeywordTok{dnorm}\NormalTok{(y,}\DecValTok{8}\NormalTok{,}\DecValTok{1}\NormalTok{)}
\end{Highlighting}
\end{Shaded}

\begin{verbatim}
## [1] 1.486720e-06 3.989423e-01 2.419707e-01
\end{verbatim}

\#9 What is a type I error? What is a p-value? How are the two
quantities related?

Type 1 error is where you falsely reject a null hypothesis that is true.
A p-value is the probability of observing a more extreme value of a test
statistic under the assumptions of the null hypothesis. The two
quantities are related because we use the p-value to conclude if we
reject or accept a null hypothesis. Sometimes we might observe an
extreme value of a test statistic by chance even when the null
hypothesis is true. This would lead to type 1 error.

\#10 What is a fundamental assumption we must make to derive inference
about regression coefficients of a linear model?

The parameters are linear.

\end{document}
